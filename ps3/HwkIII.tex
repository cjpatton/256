
\documentclass[letter]{article}
\usepackage[utf8]{inputenc}
\usepackage[margin=1in]{geometry}
\usepackage{tikz}
\usepackage{ulem}
\usepackage{graphics}
\usepackage{sidecap}
\usepackage{wrapfig}
\usepackage[toc,page]{appendix}
\usepackage{caption}
\usepackage{amssymb}
\usepackage{amsmath}
\usepackage{algorithmicx}
\usepackage{algpseudocode}

\usepackage{url,graphicx,tabularx,array,geometry,amsmath,tikz}
\usepackage{algorithm}% http://ctan.org/pkg/algorithms
\usepackage{algpseudocode}% http://ctan.org/pkg/algorithmicx
\usepackage{listings}
\usetikzlibrary{arrows}
\newenvironment{myindentpar}[1]% %indent whole paragraph when needed
 {\begin{list}{}%
         {\setlength{\leftmargin}{#1}}%
         \item[]%
 }
 {\end{list}}

\usepackage{hyperref}
\usepackage{parskip}

\hypersetup{
    colorlinks,%
    citecolor=black,%
    filecolor=black,%
    linkcolor=black,%
    urlcolor=black
}

\def\dashuline{\bgroup 
  \ifdim\ULdepth=\maxdimen  % Set depth based on font, if not set already
	  \settodepth\ULdepth{(j}\advance\ULdepth.4pt\fi
  \markoverwith{\kern.15em
	\vtop{\kern\ULdepth \hrule width .3em}%
	\kern.15em}\ULon}
\setlength\parindent{2em}

\newcounter{foot}
\setcounter{foot}{1}

\author{Olga Prilepova, Christopher Patton, Alexander Rumbaugh, \\ John Chen, Thomas Provan}

\date{\today}
\title{ECS256 - Homework II}
	
\begin{document}
\maketitle

\subsection*{Problem 1.a}
First, we'll derive $\pi_i$. The definition of the tree searching markov model leads to the 
following set of balance equations for the long-run state probabilities: 
$$ \pi_i = \pi_{i-1}q_{i-1} = \pi_0 \prod_{j=0}^{i-1}{q_j} \quad \text{ for $i\ge 1$, and } $$
$$ \pi_0 = \sum_{i=1}^\infty{\pi_i(1-q_i)} \quad \text{ for $i=0$. } $$ 
This definition for $\pi_0$ is a bit unweildy. We can also think of this quantity as one 
over the expected recurrance time, as in eq. (10.63) in the book: 
\begin{equation*}
  \begin{aligned}
         \pi_0 &= \frac{1}{E(T_{0,0})} \\ 
    E(T_{0,0}) &= 1 + \sum_{k \ne 0}{p_{0,k}E(T_{k,0})} \\ 
               &= 1 + p_{0,1}E(T_{1,0}) \\
               &= 1 + p_{0,1}(1 + \sum_{k \ne 0}{p_{1,k}E(T_{k,0})}) \\ 
               &= 1 + p_{0,1}(1 + p_{1,2}E(T_{2,0})) \\
               &= 1 + p_{0,1}(1 + p_{1,2}(1 + \sum_{k \ne 0}{p_{2,k}E(T_{k,0})})) \\
               &= 1 + p_{0,1}(1 + p_{1,2}(1 + p_{2,3}E(T_{3,0}))) 
  \end{aligned}
\end{equation*}
and so on. This unravels into a familiar closed form:   
\begin{equation*}
  \begin{aligned}
      E(T_{0,0}) &= 1 + q_0(1 + q_1(1 + q_2(1 + \dots ) \dots ))) \\ 
                 &= 1 + q_0 + q_0q_1 + q_0q_1q_2 + \dots \\
                 &= 1 + \sum_{i=1}^\infty{\big[\prod_{j=0}^{i-1}{q_j}\big]}
  \end{aligned}
\end{equation*}
If the model is positive recurrent, then there exists some value $R$ such that
$$ R = \sum_{i=1}^\infty{\big[\prod_{j=0}^{i-1}{q_j}\big]} < \infty. $$
Thus, 
$$ \pi_i = \frac{\prod_{j=0}^{i-1}{q_j}}{1 + R} \quad \text{ for $i \ge 0$. } $$


Next, $E(T_{i,0})$ follows a similar pattern. 
\begin{equation*}
  \begin{aligned}
      E(T_{i,0}) &= 1 + \sum_{k \ne 0}{p_{i,k}E(T_{k,0})} \\ 
                 &= 1 + p_{i,i+1}E(T_{j+1,0}) \\
                 &= 1 + q_i + q_iq_{i+1} + q_iq_{i+1}q_{i+2} + \dots \\
                 &= 1 + \sum_{j=i}^\infty{\Big[ \prod_{k=i}^{j}{q_k} \Big]}.
  \end{aligned}
\end{equation*}


\subsection*{Problem 1.b}
If $q_i = 0.5$ for all $i$, then $R$ is a geometric series that indeed converges. 
$$ \pi_2 = \frac{0.5 \cdot 0.5}{1 + \sum_{i=1}^\infty{0.5^{i-1}}} = 
           \frac{0.25}{1+2} \approx 0.083. $$

$$ E(T_{2,0}) = 1 + \sum_{j=2}^\infty{0.5^{j-2}} 
              = 1 + \sum_{j=1}^\infty{0.5^{j-1}} = 1 + 2 = 3. $$

\subsection*{Problem 1.c}
The rate of backtracking, in terms of the stationary probabilities $\pi_i$, is simply
$$ \sum_{i=1}^\infty{\pi_i(1 - q_i)}. $$

\end{document}
